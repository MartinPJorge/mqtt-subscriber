\documentclass{upmassignment}
\usepackage[spanish]{babel}
\usepackage{ifthen}
\usepackage{amsmath}
\usepackage{amsfonts}
\usepackage{booktabs}
\usepackage[x11names]{xcolor}
\usepackage{tcolorbox}
\usepackage{cclicenses}
\usepackage{url}
\usepackage{enumitem}

\usepackage{listings}
\lstset{basicstyle=\ttfamily,
  showstringspaces=false,
  commentstyle=\color{red},
  keywordstyle=\color{blue},
  backgroundcolor=\color{gray!30},
}


\usetikzlibrary{calc}



% Para mostrar/ocultar soluciones
\newboolean{show}
\setboolean{show}{true}
\setboolean{show}{false}
\usepackage{environ}
\NewEnviron{solucion}{
  \ifshow
      \begin{answer}\BODY\end{answer}
  \fi}






\coursetitle{Redes y Servicios}
\courselabel{RSER}
\exercisesheet{Alertas Médicas}{mini-reto: Lectura de los datos biométricos de los pacientes y desarrollo de un sistema de generación de alertas médicas}
\student{\ }%
\semester{Primer Semestre 2025/2026}
\date{\today}
\university{Universidad Politécnica de Madrid}
\school{Departamento de Ingeniería de Sistemas Telemáticos}
%\usepackage[pdftex]{graphicx}
%\usepackage{subfigure}


\setlength{\textwidth}{5.0in}
\linespread{1.3}
\renewcommand{\PB}{{\bfseries Problema}}



\begin{document}


\section*{Pregunta Esencial}
¿Cómo leer, analizar, interpretar y reaccionar en tiempo real a los datos biométricos de los pacientes para detectar alteraciones en sus constantes vitales?



\begin{figure}[h]
    \centering
    \includegraphics[width=.7\textwidth]{figs/health-monitoring}
\end{figure}
\documentclass{upmassignment}
\usepackage[spanish]{babel}
\usepackage{ifthen}
\usepackage{amsmath}
\usepackage{amsfonts}
\usepackage{booktabs}
\usepackage[x11names]{xcolor}
\usepackage{tcolorbox}
\usepackage{cclicenses}
\usepackage{url}
\usepackage{enumitem}
\usepackage{graphicx}
\usepackage{subcaption}
\usepackage{listings}
\lstset{basicstyle=\ttfamily,
  showstringspaces=false,
  commentstyle=\color{red},
  keywordstyle=\color{blue},
  backgroundcolor=\color{gray!30},
}


\usetikzlibrary{calc}



% Para mostrar/ocultar soluciones
\newboolean{show}
\setboolean{show}{true}
\setboolean{show}{false}
\usepackage{environ}
\NewEnviron{solucion}{
  \ifshow
      \begin{answer}\BODY\end{answer}
  \fi}






\coursetitle{Redes y Servicios}
\courselabel{RSER}
\exercisesheet{Alertas Médicas}{nano-reto: Lectura de los datos biométricos de los pacientes y desarrollo de un sistema de generación de alertas médicas}
\student{\ }%
\semester{Primer Semestre 2025/2026}
\date{\today}
\university{Universidad Politécnica de Madrid}
\school{Departamento de Ingeniería de Sistemas Telemáticos}
%\usepackage[pdftex]{graphicx}
%\usepackage{subfigure}


\setlength{\textwidth}{5.0in}
\linespread{1.3}
\renewcommand{\PB}{{\bfseries Problema}}



\begin{document}


\section*{Pregunta Esencial}
\emph{¿Cómo leer, analizar, interpretar y reaccionar en tiempo real a los datos biométricos de los pacientes para detectar alteraciones en sus constantes vitales?}.

Se trata de un reto social importante que se está abordando mediante soluciones tecnológicas que ya están penetrando en el mercado. Por ejemplo, EE.UU, en el año 2025, ha proporcionado a 50 millones de pacientes  un sistema de monitorización remota\footnote{https://intuitionlabs.ai/articles/remote-patient-monitoring-united-states-2025-landscape}. En la Comunidad de Madrid, también en 2025, se va a poner en marcha una plataforma para seguimiento domiciliario remoto de pacientes con diabetes, EPOC, hipertensión arterial o insfucienta cardiaca\footnote{https://www.comunidad.madrid/noticias/2025/04/23/comunidad-madrid-estrenara-este-ano-plataforma-sanitaria-inteligente-seguimiento-domiciliario-remoto-pacientes-cronicos}. Los fabricantes de smartwatches han detectado una oportunidad de negocio en este área y ya están incluyendo en sus dispositivos algoritmos de detección de insuficiencia cardíaca\footnote{https://www.androidcentral.com/wearables/samsung-galaxy-watch/samsung-galaxy-watches-will-soon-detect-warning-signs-for-heart-failure}. 

El uso de tecnologías para la monitorización remota de pacientes está reportado diversos beneficios a la sociedad. Por ejemplo, en Escocia, el uso de sistemas de monitorización de presión arterial en casa ha reducido significativamente el número de consultas médicas presenciales (400.000). Otro ejemplo es el de la Universidad de Michigan, que ha implementado un programa de monitorización remota que ha reducido las hospitalizaciones en un 59\%\footnote{https://www.thetimes.com/uk/scotland/article/home-blood-pressure-monitors-free-up-400000-gp-appointments-6q3v9kzqk}.




\begin{figure}[h]
    \centering
    % --- Primera subfigura ---
    \begin{subfigure}[b]{0.48\textwidth}
        \centering
        \includegraphics[width=\textwidth]{figs/health-monitoring}
        \label{fig:health-monitoring}
    \end{subfigure}
    \hfill
    % --- Segunda subfigura ---
    \begin{subfigure}[b]{0.42\textwidth}
        \centering
        \includegraphics[width=\textwidth]{enunciado/figs/imagefreeCC.jpg}
        \label{fig:alert-system}
    \end{subfigure}
    \label{fig:monitoring-alerts}
\end{figure}

\section*{Descripción}
En este nano-reto, usted ayudará al hospital RAID-BIO a desplegar su propio sistema de monitorización remota, utilizando sensores que registrarán de manera continua las constantes vitales de los pacientes (prácticas de laboratorio anteriores). En este nano-reto, su objetivo será desarrollar un sistema de alertas que garantice la seguridad de los pacientes. Debe proporcionar al personal sanitario una herramienta que le permita recibir alertas automáticas cada vez que se detecten valores anómalos que puedan indicar posibles emergencias.

Desarrollará el sistema de recepción y análisis de las constantes vitales utilizando el protocolo de comunicaciones MQTT. Se asumirá que los datos de los pacientes se envían al centro de datos del hospital, donde el dispositivo suscriptor recibe los datos, los evalúa y genera alertas automáticas si se superan ciertos umbrales. El especialista dispondrá de una herramienta donde visualizará las alertas generadas por los datos anómalos, como la que ha desarrollado en prácticas anteriores. En el reto de integración de la asignatura, deberá conectar el sistema de recolección de datos y alertas a la herramienta de visualización. 

Por todo lo anterior, es necesario que el sistema que va a desarrollar en este nano-reto este suscrito, a través de MQTT (script escrito en Python - \texttt{subs.py}), a los datos de los pacientes (escritos en el tópico correspondiente mediante el script \texttt{client.py}). Además, y de forma automática, este sistema deberá identificar situaciones críticas, publicando una alerta si es necesario.

En el apartado \textbf{Recursos} del nano-reto se proporcionan documentación adicional sobre esta parte del escenario de laboratorio. Consúltela antes de comenzar a abordar el reto.



\section*{Objetivos}
Los objetivos de este nano-reto es que
el estudiante sea capaz de:
\begin{itemize}
    \item Programar un \textbf{suscriptor} de MQTT que recibe y procesa datos biométricos en tiempo real, generando \textbf{alertas} automáticas cuando se detectan situaciones de riesgo.
    \item Implementar mecanismos de suscripción a un \textbf{topic}, que transporta la información biométrica de un paciente.
\end{itemize}




\section*{Actividades Guiadas}
A continuación se presentan las actividades
que debe realizar para conseguir resolver el
nano-reto:


\begin{enumerate}[label=\textbf{Actividad {{\arabic*}}}]
    \item Estudie los \textbf{Recursos} del nano-reto para entender cuáles son los componentes del escenario de laboratorio. Revise las referencias proporcionadas para el estudio del protocolo MQTT, especialmente la lógica de suscripción y recepción de mensajes. No pase a la siguiente actividad si no ha completado esta. 
        \begin{center}
            \texttt{localhost:18083}
        \end{center}
        usando como nombre de usuario \texttt{admin} y como contraseña \texttt{public}.
    \item  Verifique que el archivo (\texttt{report.csv}) ha sido generado previamente mediante la ejecución del script (\texttt{reporter.sh}) de las anteriores sesiones.
    \item \label{actividad-lectura} Implemente en \texttt{utils.py} la función
        \texttt{line\_to\_dict(fpath, line)} para disponer de un diccionario que almacena el valor asociado a cada medida de métrica biométrica (oxígeno, frecuencia cardiaca).
Al recibir la \emph{primera} línea del CSV
de las constantes vitales el resultado 
de la función debería ser:
\begin{lstlisting}[language=python]
{
  "idx": 0,
  "time": 0,
  "hr": 94,
  "resp": 21,
  "SpO2": 97,
  "temp": 36.2,
  "output": "Normal"
}
\end{lstlisting}

Para probar que la función está bien
programada ejecútela en el entorno python3:
\begin{lstlisting}
$ python3
>>> import utils
>>> utils.line_to_dict('report.csv',X+2)
\end{lstlisting}
donde $X$ es el número del grupo al que
pertenece.
Responda a la \ref{pregunta-lectura}.

    \item \label{actividad-publish} Modifique el script (\texttt{client.py}) para que, usando el diccionario
        obtenido en \texttt{line\_to\_dict(fpath, line)},  publique las constantes vitales
        \texttt{hr,resp,SpO2} y \texttt{temp}
        en topics independientes (formato: \texttt{vitals/metrica}) usando QoS~1.
        Responda a la \ref{pregunta-publish}.
    \item\label{actividad-subscribe}  Programe el script (\texttt{subs.py}) para que se suscriba exclusivamente al topic \texttt{vitals/temp} a través del broker EMQX.
        Responda a las siguientes preguntas:
        \ref{pregunta-subscribe},
      \ref{pregunta-qos},
    \ref{pregunta-puerto},
      \ref{pregunta-puerto-subs}.

      \item \label{actividad-wildcard} Programe el script (\texttt{subs.py}) para que utilice un wildcard en el topic de suscripción, permitiendo así recibir todas las métricas publicadas por (\texttt{client.py}).
        Responda a la
        \ref{pregunta-cuantas-metricas} y
        \ref{pregunta-topic-10}.

    \item
        \label{actividad-alertas}
        Modifique el script \texttt{subs.py} para que, al recibir mensajes del topic \texttt{vitals/hr}, analice si el valor recibido excede en más de un $\tfrac{X}{100}\%$ el valor medio de  $89$\,\textrm{BPM}. Si se detecta esta anomalía, el script deberá generar una alerta publicando un nuevo mensaje en el topic \texttt{alerts/hr}, con el número de BPM como contenido del mensaje. Responda a la
         \ref{pregunta-cuantas-alertas} y
         \ref{pregunta-alertas-qos}.
\end{enumerate}



\section*{Preguntas Guiadas}
A continuación, se presentan las preguntas
guiadas asociadas a cada una de las
actividades guiadas de la práctica.
Respóndalas para preparar el material
de entrega que servirá para evaluar cómo
se ha enfrentado a este nano-reto:
\begin{enumerate}[label={\bf Pregunta \arabic*}]
    \item \label{pregunta-lectura}
        Guarde el resultado
        de la
        \ref{actividad-lectura}
        en el campo
        \texttt{lectura} del JSON
        de respuestas.
    \item \label{pregunta-publish}
        Tras hacer la \ref{actividad-publish}
        inicie una
        captura en Wireshark en la interfaz
        \texttt{lo}. 
        Ejecute el \texttt{client.py} para que reporte las constantes
        al broker EMQX\footnote{Recuerde identificar
        el puerto y dirección usados por EMQX para
        las comunicaciones MQTT.}.
        Deje que se publiquen
        varias constantes vitales y guarde\footnote{
        Recuerde que todas las capturas
        de la práctica deben contener
        \emph{solo} los paquetes MQTT.
}
        en\\
        \texttt{publishes-grupoX.pcapng}
        la captura.

        ¿Cuál es el máximo valor
        de \texttt{temp} reportado en la
        captura? Responda en el campo
        \texttt{maxtemp} del JSON de respuestas.

    \item\label{pregunta-subscribe}
        Tras realizar la
        \ref{actividad-subscribe}
        inicie una captura Wireshark
        en la interfaz \texttt{lo}, después
        ejecture
        \texttt{client.py} y
        \texttt{sub.py}.
        Guarde la captura en\\
        \texttt{subs-temp-grupoX.pcapng}.
        ¿Cuál es el tipo de mensaje
      del\\
      \texttt{Subscribe request}?
    Responda a la pregunta en el campo
    \texttt{reqtype}
    del JSON de respuestas.

      \item\label{pregunta-qos} ¿Con qué QoS se suscribe
      al topic \texttt{vitals/temp}?
    Responda a la pregunta en el campo
    \texttt{subsqos} del JSON de respuestas.

    \item\label{pregunta-puerto} ¿Cuál es el puerto
      utilizado por el
      cliente para publicar mensajes?
    Responda a la pregunta en el campo
    \texttt{pubport} del JSON de respuestas.

    \item\label{pregunta-puerto-subs} ¿Cuál es el puerto
        utilizado por el
        suscriptor para recibir mensajes?
    Responda a la pregunta en el campo
    \texttt{subport} del JSON de respuestas.

    \item\label{pregunta-cuantas-metricas}
        Tras realizar la
        \ref{actividad-wildcard}
        inicie una captura Wireshark
        y ejecute el \texttt{client.py} y
        \texttt{subs.py}. Deje que se publiquen
        unos veinte \texttt{PUBLISH} y guarde
        en\\
        \texttt{subs-wildcard-grupoX.pcapng}
        la captura de Wireshark.
        ¿A cuántas métricas se suscribe el \texttt{subs.py}?
        Responda en el campo
        \texttt{nummetricas} del JSON
        de respuestas.
    \item\label{pregunta-topic-10}
        ¿Cuál es el topic del
        \texttt{PUBACK} con
        \texttt{Message identifier} $10+X$?
        Responda en el campo
        \texttt{pubacktopic} del JSON
        de respuestas.

    \item\label{pregunta-cuantas-alertas}
        Tras realizar la
        \ref{actividad-alertas}
        inicie una captura de tráfico en Wireshark sobre la interfaz \texttt{lo}.
        Ejecute \texttt{client.py} y
        \texttt{subs.py} y espere a que se
        detecten y publiquen al menos un
        par de alertas.
        Guarde la captura resultante con el
        nombre \texttt{alertas-grupoX.pcapng}.
        ¿Cuántas alertas se envían?
        Responda en el campo \texttt{numalerts}
        del JSON de respuestas.


    \item\label{pregunta-alertas-qos}
        ¿Cuál es la QoS con la que se envían
        las alertas?
        Responda en el campo \texttt{qosalerts}
        del JSON de respuestas.
\end{enumerate}



\section*{Recursos}
\subsection*{Documentación teórica}
\begin{enumerate}
    \item Guía básica de Python 3 (disponible en: \url{http://python.org/doc}).
    \item Manual de uso de Wireshark para análisis de protocolos. (disponible en: \url{https://www.wireshark.org/docs/wsug_html_chunked/}).
    \item Documentación del protocolo MQTT y funcionamiento de los niveles de QoS (disponible en: \url{https://www.paessler.com/es/it-explained/mqtt}).
\end{enumerate}


\subsection*{Documentación para la implementación}
\begin{enumerate}
    \item Descripción del escenario de
        alertas médicas (disponible
        en Moodle).
    \item Broker MQTT (broker.emqx.io) (disponible en: \url{https://docs.datadoghq.com/es/integrations/emqx/})
    \item Published/Subscriber MQTT con EMQX: \url{https://www.emqx.com/en/blog/how-to-use-mqtt-in-python}
\end{enumerate}



\section*{Entrega}
\noindent Se subirá a moodle un archivo
\texttt{subscriberX.zip}
(con \texttt{X} el número de grupo)
que contenga:
\begin{enumerate}
    \item el cliente \texttt{client.py};
    \item el cliente \texttt{subs.py};
    \item el archivo \texttt{utils.py};
    \item el JSON de respuestas \texttt{respuestas-X.json}; y
    \item las trazas de Wireskark
        \texttt{publishes-grupoX.pcapng},\
        \texttt{subs-temp-grupoX.pcapng},\
        \texttt{subs-wildcard-grupoX.pcapng},\
        \texttt{alertas-grupoX.pcapng}
\end{enumerate}
 
\begin{tcolorbox}
    \textbf{Atención I}: las capturas
    deben contener
    \emph{solamente} tráfico MQTT.\\
    \textbf{Atención II}: una entrega
    sin los archivos especificados,
    o con archivos sin formato especificado
    tendrá un 0 en las \textbf{Preguntas}
    correspondientes.
\end{tcolorbox}


\end{document}
