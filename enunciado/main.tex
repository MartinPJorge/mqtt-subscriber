\documentclass{upmassignment}
\usepackage[spanish]{babel}
\usepackage{ifthen}
\usepackage{amsmath}
\usepackage{amsfonts}
\usepackage{booktabs}
\usepackage[x11names]{xcolor}
\usepackage{tcolorbox}
\usepackage{cclicenses}
\usepackage{url}

\usepackage{listings}
\lstset{basicstyle=\ttfamily,
  showstringspaces=false,
  commentstyle=\color{red},
  keywordstyle=\color{blue},
  backgroundcolor=\color{gray!30},
}


\usetikzlibrary{calc}



% Para mostrar/ocultar soluciones
\newboolean{show}
\setboolean{show}{true}
\setboolean{show}{false}
\usepackage{environ}
\NewEnviron{solucion}{
  \ifshow
      \begin{answer}\BODY\end{answer}
  \fi}






\coursetitle{Redes y Servicios}
\courselabel{RSER}
\exercisesheet{Suscriptor MQTT}{}
\student{\ }%
\semester{Primer Semestre 2024/2025}
\date{\today}
\university{Universidad Politécnica de Madrid}
\school{Departamento de Ingeniería de Sistemas Telemáticos}
%\usepackage[pdftex]{graphicx}
%\usepackage{subfigure}


\setlength{\textwidth}{5.0in}
\linespread{1.3}
\renewcommand{\PB}{{\bfseries Problema}}



\begin{document}

\section*{Introducción}

\noindent
En esta práctica\footnote{Todas las preguntas
tienen el mismo valor/puntuación.
}\footnote{Este material está protegido por la
licencia
CC BY-NC-SA 4.0.
\byncsa
} vamos a programar un
suscriptor MQTT \texttt{subs.py}
que recolecte las métricas
publicadas por el cliente \texttt{client.py}.
Para ello
abriremos una suscripción con el broker
y nos suscribiremos a los topics de
reporte de métricas. Finalmente,
haremos un detector de anomalías en
el pulso para publicar alertas desde
el \texttt{subs.py}.
La figura mostrada a continuación resalta
con fondo oscuro las componentes que
modificaremos:
\begin{figure}[h]
    \centering
\begin{tikzpicture}
    \node[draw,rectangle] (sensor) at (0,0)
        {sensor};
    \node[draw,rectangle] (script)
        at ($(sensor)+(0,-1)$)
        {\texttt{reporter.sh}};
    \node[draw,rectangle,anchor=west]
        (report)
        at ($(script.east)+(1,0)$)
        {\texttt{report.csv}};
    \node[draw,rectangle,fill=gray!30]
        (utils)
        at ($(report)+(0,1)$)
        {\texttt{utils.py}};
    \node[draw,rectangle,fill=gray!30]
        (client)
        at ($(utils)+(0,1)$)
        {\texttt{client.py}};
    \node[draw,rectangle,fill=gray!30,
        align=center]
        (server)
        at ($(client)+(2.5,0)$)
        {broker\\emqx};
    \node[draw,rectangle,fill=gray!30]
        (datasource)
        at ($(server)+(3,0)$)
        {\texttt{subs.py}};

    \draw[arrows=<->] (sensor.south)
        -- (script.north);
    \draw[arrows=->] (script.east)
        -- (report.west);
    \draw[arrows=<->] (report.north)
        -- (utils.south);
    \draw[arrows=<->] (utils.north)
        -- (client.south);
    \draw[arrows=<->] (client.east)
        --
        node[midway,above]
        {pub}
        (server.west);
    \draw[arrows=<->] (server.east)
        --
        node[midway,above]
        {sub/pub}
        (datasource.west);



\end{tikzpicture}
\caption{escenario de la práctica.}
\label{fig:escenario}
\end{figure}




\section*{Arrancar broker EMQX}
\noindent
Antes de comenzar a programar el
\texttt{subscriber.py} tenemos que
arrancar el broker EMQX. Para ello siga
los pasos de la práctica anterior.
En caso de que el contenedor del broker
EMQX ya esté disponible, ejecute el
siguiente comando para arrancarlo de nuevo:
\begin{lstlisting}[language=bash]
$ docker start `docker ps | grep "emqx:5.0.2"\
  | cut -d' ' -f1`
\end{lstlisting}
Puede comprobar que el broker se está
ejecutando correctamente accediendo al
panel de EMQX desde el navegador:
\begin{center}
    \url{localhost:18083}
\end{center}
usando el como nombre de usuario
\texttt{admin} y como contraseña
\texttt{public}.



\section*{Publicar métricas en topics}
Tras arrancar el broker EMQX vamos a
publicar cada métrica en un topic distinto.
En concreto, publicaremos cada columna
del \texttt{report.csv} en el topic
\texttt{vitals/metrica}. Por ejemplo,
el ritmo cardíaco \texttt{hr} se publica
en el topic \texttt{vitals/hr}.

\subsection*{Crear diccionario de métricas}
Para ello debe modificar crear una
función en \texttt{utils.py} que reciba
una línea de fichero y devuelva un diccionario
python. Llame a dicha función
\texttt{line\_to\_dict(fpath,line)} y prográmela
para que devuelva el siguiente diccionario
al recibir la primera línea del CSV
de las constantes vitales:
\begin{lstlisting}[language=python]
{
  "idx": 0,
  "time": 0,
  "hr": 94,
  "resp": 21,
  "SpO2": 97,
  "temp": 36.2,
  "output": "Normal"
}
\end{lstlisting}

Para probar que la función está bien
programada ejecútela en el entorno python3:
\begin{lstlisting}
$ python3
>>> import utils
>>> utils.line_to_dict('report.csv',X+2)
\end{lstlisting}
donde $X$ es el número del grupo al que
pertenece.

\begin{problemlist}
    \pbitem Guarde el resultado en el campo
        \texttt{lectura} del JSON
        de respuestas.
\end{problemlist}


\subsection*{Publicar en distintos topics}
A continuación vamos a publicar cada
métrica de las constantes vitales en un
topic diferente. En concreto, vamos a publicar
las constantes vitales
\texttt{hr},
\texttt{resp},
\texttt{SpO2} y
\texttt{temp}.
Para ello va a modificar el código de su
\texttt{client.py} para que haga lo siguiente
cada vez que lee una línea de constantes
vitales:
\begin{enumerate}
    \item obtener el diccionario de la línea
        usando
        \texttt{line\_to\_dict(fpath,line)}; y
    \item publicar cada constante vital
        el un topic usando QoS~1.
\end{enumerate}


Una vez haya modificado en \texttt{client.py},
ejecútelo para que reporte las constantes
al broker EMQX\footnote{Recuerde identificar
el puerto y dirección usados por EMQX para
las comunicaciones MQTT.}, inicie una
captura en Wireshark en la interfaz
\texttt{lo}. Deje que se publiquen
varias constantes vitales y guarde\footnote{
Recuerde que todas las capturas
de la práctica deben contener
\emph{solo} los paquetes MQTT.
}
en \texttt{publishes-grupoX.pcapng}
la captura.

\begin{problemlist}
    \stepcounter{enumi}
    \pbitem ¿Cuál es el máximo valor
        de \texttt{temp} reportado en la
        captura? Responda en el campo
        \texttt{maxtemp} del JSON de respuestas.
\end{problemlist}



\section*{Suscripción a topic de temperatura}
A continuación va a suscribirse al
topic \texttt{vitals/temp} usando el
script \texttt{subs.py} que se proporciona
en la práctica. Este script lo usaremos
para suscribirnos al topic de temperatura
a través del broker EMQX.

Una vez haya modificado el script
\texttt{subs.py}, inicie una captura
wireshark. Después ejecute
el \texttt{client.py} y \texttt{subs.py}
para comprobar que se reciben las métricas
de temperatura. 
Detenga la captura cuando se hayan
recibido un par de métricas de temperatura
y guarde la captura en el
fichero \texttt{subs-temp-grupoX.pcapng}.

Fíjese en la captura de wireshark y
responda a las siguientes preguntas:
\begin{problemlist}
    \stepcounter{enumi}
    \stepcounter{enumi}
    \pbitem ¿Cuál es el tipo de mensaje
        del \texttt{Subscribe request}?
    \pbitem ¿Con qué QoS se suscribe
        al topic \texttt{vitals/temp}?
    \pbitem ¿Cuál es el puerto
        utilizado por el
        cliente para publicar mensajes?
    \pbitem ¿Cuál es el puerto
        utilizado por el
        suscriptor para recibir mensajes?
\end{problemlist}
Responda a las preguntas en los
siguientes campos del JSON de respuestas:
\texttt{reqtype},
\texttt{subsqos},
\texttt{pubport} y
\texttt{subport}; respetivamente





\section*{Suscripción a varios topics}
Ahora vamos a suscribirnos a todos los
topics de constantes vitales que reporte
nuestro \texttt{client.py}. Para ello,
use un wildcard en el topic al que se
suscribe el \texttt{subs.py} y compruebe
que el suscriptor recibe todas las métricas
reportadas por el \texttt{client.py}.


Inicie una captura wireshark
y ejecute el \texttt{client.py} y
\texttt{subs.py}. Deje que se publiquen
unos veinte \texttt{PUBLISH} y guarde en
\texttt{subs-wildcard-grupoX.pcapng}
la captura de wireshark.
\begin{problemlist}
    \stepcounter{enumi}
    \stepcounter{enumi}
    \stepcounter{enumi}
    \stepcounter{enumi}
    \stepcounter{enumi}
    \stepcounter{enumi}
    \pbitem ¿A cuántas métricas se suscribe
        el \texttt{subs.py}?
    \pbitem ¿Cuál es el topic del
        \texttt{PUBACK} con
        \texttt{Message identifier} $10+X$?
\end{problemlist}
Responda a las preguntas en los
siguientes campos del JSON de respuestas
\texttt{nummetricas} y
\texttt{pubacktopic}.




\section*{Alerta por anomalías}
Para terminar vamos a generar alertas cuando
la métrica \texttt{hr} exceda en un
$\tfrac{X}{100}$ por ciento el valor medio
de 89\,\textrm{BPM}.
Para ello debe modificar el script
\texttt{subs.py} para que:
\begin{enumerate}
    \item compruebe que recibe un mensaje
        del topic \texttt{vitals/hr}; y
    \item envíe un \texttt{PUBLISH} en caso
        de que se cumpla la condición
        para generar la alerta.
\end{enumerate}
Cuando publique la alerta, hágalo en
el topic \texttt{alerts/hr} y envíe como
mensaje cuáles son los BPM.

Una vez tenga funcionando el script
\texttt{subs.py} debe iniciar una captura
wireshark en la interfaz \texttt{lo}.
Después inicie el \texttt{client.py}
y el \texttt{subs.py}. Espere a que se
publiquen un par de alertas antes de parar
la captura, guárdela en el archivo
\texttt{alertas-grupoX.pcapng}, y responda
a las siguientes preguntas:
\begin{problemlist}
    \stepcounter{enumi}
    \stepcounter{enumi}
    \stepcounter{enumi}
    \stepcounter{enumi}
    \stepcounter{enumi}
    \stepcounter{enumi}
    \stepcounter{enumi}
    \stepcounter{enumi}
    \pbitem ¿Cuántas alertas se envían?
    \pbitem ¿Cuál es la QoS con la que envía
        las alertas?
\end{problemlist}
Responda a las preguntas en los
siguientes campos del JSON de respuestas:
\texttt{numalerts} y
\texttt{qosalerts}.




\section*{Entrega}
\noindent Se subirá a moodle un archivo
\texttt{subscriberX.zip}
(con \texttt{X} el número de grupo)
que contenga:
\begin{enumerate}
    \item el cliente \texttt{client.py};
    \item el cliente \texttt{subs.py};
    \item el archivo \texttt{utils.py};
    \item el JSON de respuestas \texttt{respuestas-X.json}; y
    \item las trazas de Wireskark
        \texttt{publishes-grupoX.pcapng},\
        \texttt{subs-temp-grupoX.pcapng},\
        \texttt{subs-wildcard-grupoX.pcapng},\
        \texttt{alertas-grupoX.pcapng}
\end{enumerate}

\begin{tcolorbox}
    \textbf{Atención I}: las capturas
    deben contener
    \emph{solamente} tráfico MQTT.\\
    \textbf{Atención II}: una entrega
    sin los archivos especificados,
    o con archivos sin formato especificado
    tendrá un 0 en los \textbf{Problemas}
    correspondientes.
\end{tcolorbox}


\end{document}
